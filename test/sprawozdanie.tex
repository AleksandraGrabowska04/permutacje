
\documentclass{article}
\usepackage[dvipsnames]{xcolor}

\usepackage{fancyvrb}

\RecustomVerbatimCommand{\VerbatimInput}{VerbatimInput}
{fontsize=\footnotesize, frame=lines, framesep=2em, rulecolor=\color{Gray}, label=\fbox{\color{Black}godzinaRozpoczecia.txt}, labelposition=topline, commandchars=\|\(\), commentchar=*}


\usepackage[dvipsnames]{xcolor}
\usepackage[T1]{fontenc}

	\addtolength{\oddsidemargin}{-.875in}
	\addtolength{\evensidemargin}{-.875in}
	\addtolength{\textwidth}{1.75in}

	\addtolength{\topmargin}{-.875in}
	\addtolength{\textheight}{1.75in}
\begin{document}
\begin{large}

\section{Wstęp}
Program testowy do sprawdzania czasu wykonywania programu do permutacji.
\\Godzina rospoczęcia i zakończenia działania programu test.sh:
\begin{center}
\begin{figure}[hbt!]
\centering
\VerbatimInput{./test/godzinaRozpoczecia.txt}
\end{figure}
\end{center}

Lokalizacje plików:

\begin{table}[hbt!]

\begin{tabular}{|l|l|l|} 
\hline
Nazwa pliku & Funckja & Lokalizacja   \\
\hline
main   & Program liczący permutacje.   & ./build  \\
run.sh  & Skrypt uruchamiający main, tworzy plik pdf z wynikami programu & Katalog głowny. \\
compile.sh & Kompiluje program. & Katalog głowny.    \\
test.sh   & Sprawdza czas wykonywania programu dla różnych zmiennych.  & Katalog głowny. \\  
godzinaRozpoczecia.txt  & Przechowuje godzine rozpoczęcia i zakończenia test.sh. & ./test\\
\hline 
\end{tabular}
\end{table}

\IfFileExists{./test/wynikTestA.txt}{
\section{Wyniki Testu typu A.}
\begin{center}
\RecustomVerbatimCommand{\VerbatimInput}{VerbatimInput}
{fontsize=\footnotesize, frame=lines, framesep=2em, rulecolor=\color{Gray}, label=\fbox{\color{Black}wynikTestA.txt}, labelposition=topline, commandchars=\|\(\), commentchar=*}
\begin{figure}[hbt!]
\centering
\VerbatimInput{./test/wynikTestA.txt}
\end{figure}
\end{center}}

\IfFileExists{./test/wynikTestB.txt}{
\section{Wyniki Testu typu B.}
\begin{center}
\RecustomVerbatimCommand{\VerbatimInput}{VerbatimInput}
{fontsize=\footnotesize, frame=lines, framesep=2em, rulecolor=\color{Gray}, label=\fbox{\color{Black}wynikTestB.txt}, labelposition=topline, commandchars=\|\(\), commentchar=*}
\begin{figure}[hbt!]
\centering
\VerbatimInput{./test/wynikTestB.txt}
\end{figure}
\end{center}
\pagebreak}

\IfFileExists{./test/wynikTestC.txt}{
\section{Wyniki Testu typu C.}
\RecustomVerbatimCommand{\VerbatimInput}{VerbatimInput}
{fontsize=\footnotesize, frame=lines, framesep=2em, rulecolor=\color{Gray}, label=\fbox{\color{Black}wynikTestC.txt}, labelposition=topline, commandchars=\|\(\), commentchar=*}
\begin{figure}[hbt!]
\centering
\VerbatimInput{./test/wynikTestC.txt}
\end{figure}}


\end{large}

\end{document}

